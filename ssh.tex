% SPDX-FileCopyrightText: 2006-2024 Knut Reinert & Freie Universität Berlin
% SPDX-FileCopyrightText: 2016-2024 Knut Reinert & MPI für molekulare Genetik
% SPDX-License-Identifier: CC-BY-4.0

\documentclass{article}
\usepackage[utf8]{inputenc}
\usepackage[margin=1in]{geometry} % Reduced margin
% lmodern + fontenc make ~ align to the middle
\usepackage{lmodern}
\usepackage[T1]{fontenc}

\title{Getting Started with SSH}
\date{}

\setlength{\parindent}{0pt}

\begin{document}

\maketitle

\section{Introduction}
SSH (Secure Shell) is a cryptographic network protocol for operating network services securely over an unsecured network. It provides a secure channel over an unsecured network by encrypting the communication between the client and server.

\section{ZEDAT Account}

You need a ZEDAT account that is registered with the Fachbereich. To make sure your account is activated, please log in into \verb|https://portal.mi.fu-berlin.de| once. You may need to wait up to an hour.

\section{Installing SSH Client}

\begin{description}
    \item[Linux:] ssh should be installed. If not, run: \verb|sudo apt install openssh-client|
    \item[Mac:] ssh should be installed.
    \item[Windows:] Install WSL (Windows Subsystem for Linux). ssh should then be available in WSL.
\end{description}

\section{Setting Up Your ssh Config}

In your terminal, type:
\begin{verbatim}
    mkdir -p ~/.ssh
    touch ~/.ssh/config
    chmod 600 ~/.ssh/config
    nano ~/.ssh/config
\end{verbatim}

This will open a file. Now copy the following into this file and replace \verb|<USER>| with your ZEDAT username.

\begin{verbatim}
Host andorra
Hostname andorra.imp.fu-berlin.de
User <USER>

# The following uses andorra as a proxy to let you access
# all compute servers and cuda servers.
# For example, `ssh compute01`
Host compute?? cuda??
HostName %h.imp.fu-berlin.de
ProxyJump andorra
User <USER>
\end{verbatim}

Save and exit \verb|nano| (\texttt{<ctrl-o>}, \texttt{<enter>}, \texttt{<ctrl-x>}).

\section{Connecting to the FU Server}

After setting up your config, type:

\begin{verbatim}
    ssh compute01
\end{verbatim}

If you are connecting for the first time, you will see a message asking you whether you trust the server. Type \texttt{yes} and press \texttt{<enter>} to continue connecting.

You are also asked for your ZEDAT password. Maybe even multiple times.

\section{Connecting to the Servers with an ssh Key Pair}

There are still two issues left to solve:

\begin{enumerate}
    \item You always have to type in your password (that's a waste of time)
    \item \textbf{You cannot log in if you are not connected to eduroam}.
\end{enumerate}

To overcome these issues, we need to generate an ssh key pair.

\subsection{Checking whether you already have an ssh key pair}

Type the following in your terminal:

\begin{verbatim}
    ls ~/.ssh
\end{verbatim}

It should print at least the config file we created, and might show this:

\begin{verbatim}
    config   id_rsa  id_rsa.pub
\end{verbatim}

If you see a file that ends with \verb|.pub|, you already have an ssh key that you can reuse.

\textbf{If you already have an ssh key, SKIP 6.2!}

\subsection{Creating an ssh key}

Type into your terminal:

\begin{verbatim}
    ssh-keygen -t rsa -b 4096 -C "your_email@example.com"
\end{verbatim}

The string after \texttt{-C} is a comment for your key that you can freely choose.

If you are asked for a passphrase, just press \texttt{<enter>}. Otherwise, you will have to enter your ssh key password each time you want to use it.

\subsection{Copy the ssh-key to the server}

Type into your terminal:

\begin{verbatim}
    ssh-copy-id andorra
\end{verbatim}

Type in your ZEDAT password if requested to do so.

\section{Done}

You should now be able to log in to the servers from everywhere without entering your password every time.
\end{document}

