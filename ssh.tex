\documentclass{article}
\usepackage[utf8]{inputenc}
\usepackage[margin=1in]{geometry} % Reduced margin

\title{Getting Started with SSH}
\date{}

\setlength{\parindent}{0pt}

\begin{document}
	
	\maketitle
	
	\section{Introduction}
	SSH (Secure Shell) is a cryptographic network protocol for operating network services securely over an unsecured network. It provides a secure channel over an unsecured network by encrypting the communication between the client and server. 
	
	\section{Zedat Account}
	
	You need a Zedat account. To make sure it is registered/activated, please login once into \verb|https://portal.mi.fu-berlin.de|. You may need to wait up to an hour.
	
	\section{Installing SSH Client}
	
	\begin{description}
		\item[Linux:] ssh should be installed. Otherwise do: \begin{verbatim}
			sudo apt-get install openssh-client
		\end{verbatim}
		\item[Mac:] ssh should be installed
		\item[Windows: ] Install WSL (WIndows Subsystem for Linux). ssh should then be available
	\end{description}
	
	\section{Set up you ssh config file}
	
	In your terminal, type:
	\begin{verbatim}
		nano ~/.ssh/config
	\end{verbatim}

	This will open a file. Now copy the following into this file and change \verb|<USER>| with you Zedat username.
	
	\begin{verbatim}
Host andorra
Hostname andorra.imp.fu-berlin.de
User <USER>

# The following proxy jumps over andorra and letts you access
# all compute servers and cudo servers
# e.g. ssh compute01
Host compute?? cuda??
HostName %h.imp.fu-berlin.de
ProxyJump andorra
User <USER>

	\end{verbatim}
	
	Save and Exit \verb|nano|.
	
	\section{Connect to the FU server}
	
	Now that you have your config set up. Type:
	
	\begin{verbatim}
		ssh compute01
	\end{verbatim}
	
	If you're connecting for the first time, you may see a message asking you to verify the authenticity of the host. Type \texttt{yes} to continue connecting.
	
	You are also asked for your Zedat password. Maybe even multiple times.
	
	\section{Connect to the servers with an ssh key pair}
	
	With the above there are two issues:
	
	\begin{enumerate}
		\item You always have to type in your password (thats a waste of time)
		\item You \textbf{cannot login if you are not connected to eduroam}.	
	\end{enumerate}
	
	To overcome these issues, we need to generate a ssh key pair.
	
	\subsection{Check if you already have an ssh key pair}
	
	Type the following in your terminal:
	
	\begin{verbatim}
		ls ~/.ssh
	\end{verbatim}
	
	It should print at least the config file we created. And might show this:
	
	\begin{verbatim}
		 config   id_rsa  id_rsa.pub
	\end{verbatim}
	
	If you see a file that ends with \verb|.pub| you already have and ssh and you can reuse it.
	
	\textbf{If you already have an ssh key SKIP 6.2!}
	
	\subsection{Create an ssh key}
	
	Type into you terminal:
	
	\begin{verbatim}
		ssh-keygen -t rsa -b 4096 -C "your_email@example.com"
	\end{verbatim}
	
	If you are asked for a pass phrase, just press Enter (we recommend not setting a password).
	
	\subsection{Copy the ssh-key to the server}
	
	Type into you terminal:
	
	\begin{verbatim}
		ssh-copy-id andorra
	\end{verbatim}

	Type in your zedat password if asked for.
	
	\section{Done}
	
	You should now be able to login to the servers from everywhere without adding your password everytime.
\end{document}

